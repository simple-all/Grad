\section{Cost and Risk}

The cost and risk associated with actively changing an SoS architecture is not a trivial problem. Obvious costs and risks, such as the monetary costs of implementing a new architecture or the risk of failing to predict the future environment, are similarly encountered in system-level engineering. The more unique aspects of cost and risk in a System of Systems in outlined below.

A significant property of an SoS is that it exhibits emergent behavior, or properties that are not part of, or apparent, in individual systems \cite{Sage2001,DeLaurentis2005}. As such, predicting the performance of a new SoS architecture is inherently difficult and carries risk when considering implementing a new architecture. Mitigating this kind of risk, or emergent risk, requires a firm grasp on SoS dynamics, accurate models of the SoS, and a reliable prediction of the future environment.

Cost can appear in multiple forms in an adaptive SoS. The need for an adaptive architecture assumes that the SoS is not operating at full potential. Therefore, if implementing a new architecture will increase the ability of the SoS to achieve its common goal, there may be reduction in the cost of not fully meeting the common goal. The inability to adapt may also pose increased costs to an SoS. Given that an SoS will need to meet a performance target for a number of possible scenarios, some level of capability is required for each scenario. This could be met by either investing in new or upgraded systems to add to the collective, but if an adaptive approach could be used instead, a smaller set of collaborating systems may require less investment.