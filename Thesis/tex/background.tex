\section{Background}

Adaptability of an SoS has largely been focused on the evolution of an SoS over time when acquiring new systems, not actively modifying the architecture in response to a changing environment. However, many of the principles applied to the evolution of an SoS over time still provide value and insight into adaptability if the BMDS is framed as an acknowledged SoS.

An acknowledged SoS is characterized as a System of Systems sharing common objectives, management, and authority, but systems still retain their own management and authority in unison with the overarching SoS goals \cite{Agarwal2014, Agarwal2015, Zhemei2018}. There has been significant effort to characterize such systems and their evolution over time, as well as proposed methods to better optimize how such systems can negotiate and plan for optimal capability. (TODO here: Dynamic optimization, trade contracts, MUSTDO, etc). Regardless of the method, a fully realized plan for the evolution of a system requires that systems with authority acknowledge possible risks and plan for mitigating those risks \cite{Zhemei2018}. 

In contrast to the wealth of literature on managing evolutions of an SoS on a large time scale, relatively few papers have been published investigating how an existing SoS may adapt itself to better fit a changing environment (Are there any?...). Adaptability is characterized as the ability of an SoS to react when environmental factors reduce the SoS's performance, thereby regaining some of the lost performance \cite{Ackoff1971}. In contrast, evolution of an SoS takes place on long time scales with deliberate action from the SoS authorities in order to increase capability in its expected environment \cite{}.

Some system-level approaches exhibit similar problems to an SoS. For example, characterizing the performance of alternate configurations of subsystems mirrors many challenges faced by characterizing the performance of different SoS architectures. Optimization through adaptation of a system simply requires the ability to adjust component parameters \cite{DeRoo2009}. Haris et. al. proposed a method for using a combination of fuzzy logic, quality function deployment, and genetic algorithms to characterize the available performance space of subsystems in relation to customer goals, risks, and budget \cite{Haris2011}. While such approaches do show success in analyzing alternative configurations of subsystems, they fail to address critical properties that separate an SoS from a simple system. Systems-level approaches characterize a single assemblage of components which act in unison to perform a function, whereas an SoS is an assemblage of systems that share a common goal \cite{Maier1996}. In consequence, systems-level adaptivity or optimization approaches do not address how components can be reconfigured or reorganized in order to improve performance, but rather target the modification of component parameters or the selection of a subsystem.